%%
% This file is a sample of the main body of a dissertation
% at the Graduate School of Systems and Information Engineering,
% University of Tsukuba.
% You can rewrite this file to make a thesis body
% with the same format as this example using LaTeX.
% Depending on your PC environment and
% the settings of your LaTeX environment,
% you may need to change the kanji code and line feed code.
%%

\documentclass[12pt, a4paper]{report}
%\usepackage[utf8]{inputenc}

% IMPORTANT
\usepackage{sie-en}

\usepackage{graphicx} 
\usepackage{times}
\usepackage{xurl}
\usepackage[scale=0.85]{noto-mono}

\setcounter{tocdepth}{3}
\setcounter{page}{-1}

\title{The Example of Thesis \\ at the Graduate School of \\ Systems and Information Engineering}

\author{Luca Foppiano}

\programfield{Master’s Program in Computer Science}

\advisor{Daijiro Tsukuba}

%% Name of department + year and month
%% Please rewrite as necessary.
\majorfield{the Degree of Master of XXXXXX}

\graduateyear{20XX}
\graduatemonth{March}

\abstract{
    \noindent
    This is a sample document for writing a thesis which will be submitted to the Degree
    Programs in Systems and Information. 
    %
    The main purpose of this document is to reduce the author's trouble in making title.
    %
    Of course you can create your title page from scratch. 
    %
    But your style must fulfill the requirements such as margins.
}
%%%%%

\begin{document}

\maketitle
\makeabstract
\maketableofcontents

%\pagenumbering{roman} % I, II, III, IV 
%{
%  \setlength{\parskip}{0pt}
%  \tableofcontents
%  \listoffigures
%  \listoftables
%}
%\pagebreak \setcounter{page}{1}
%\pagenumbering{arabic} % 1,2,3

\chapter{Introduction}

This is a sample document for writing a thesis which will be submitted to the Degree
Programs in Systems and Information. The main purpose of this document is to reduce
the author's trouble in making title. Of course you can create your title page from
scratch. But your style must fulfill the requirements such as margins.

This document is also prepared for intention of unifying the thesis style. You can
write your thesis based on this document. This document does not include how to
write good thesis. So, you should consult advisor(s) about the concrete example of thesis
and other details.

\chapter{Format}

This chapter describes how to write the cover and body of the paper.

\section{Cover Page}

The cover page is created by {\tt $\backslash$maketitle}, which should
contain a string corresponding to each of the following items.


\begin{description} \parskip=1pt
\item{Title:}
  The title should appear in {\tt $\backslash$title}.  To change the line, enter $\backslash\backslash$.
  %
  If the title is more than four lines long, the cover page will be overflowed and the style file ``sie-en.sty'' will have to be changed.
\item{Author:}
  Name of author should be entered in {\tt $\backslash$author}.
\item{Advisor:}
  Name of your advisor should be entered in {\tt $\backslash$advisor}.
\item{Major Field:}
  Name of your major field should be entered in {\tt $\backslash$majorfield}.
\item{Degree Program:}
  Name of degree program should be entered in {\tt $\backslash$programfield}.
\item{Date:}
  Month and year of completion should be entered in {\tt $\backslash$graduatemonth} and {\tt $\backslash$graduateyear}, respectively. 
\end{description}

\section{Main body}

The body of the text should be in one column.

Tables and figures are numbered, captioned and referred to in the
text. Tables \ref{table:fundamental_data_type} and figure
ref{figure:sample} are examples of tables and figures
respectively.
%
The description of the table is often written above and the
description of the figure below.
%
The package used to insert the figure The package used to insert the
figure should be chosen freely according to the environment in which
it is used.

\begin{table}[hbt]
\caption{Sample of Table}
\label{table:fundamental_data_type}
\begin{center}
\begin{tabular}{| c | r | r | r | r |}
\hline
Fiscal Year & 1st & 2nd & 3rd & 4th \\
\hline
1995 & 85 & 92 & 86 & 88 \\
1996 & 83 & 89 & 90 & 102 \\
1997 & 88 & 87 & 91 & 112 \\
1998 & 144 & 93 & 90 & 115 \\
\hline 
\end{tabular}
\end{center}
\end{table}
\medskip

\begin{figure}[htbp]
\begin{center}
\includegraphics[width=3cm]{sample.eps}
%\psfig{file=sample.eps,scale=0.6}
%\epsfile{file=sample.eps,scale=0.6}
\end{center}
\caption{Example of figure}
\label{figure:sample}
\end{figure}

For more information, see the (slightly outdated)
\cite{rakuraku,jiyu-jizai} reference book. Also, see Haruhiko
Okumura's ``Japanese TeX FAQ'' at 
\url{http://www.matsusaka-u.ac.jp/~okumura/texfaq/} is a good source of
information on \TeX in Japanese. It is a good source of information on
Japanese TeX. The following are examples of references to specific
papers \cite{bryant-ieeetc86}.

\chapter{Automatic identification and normalization of physical quantities from scientific publications}


\chapter{SuperMat: a linked dataset }

\chapter{Automatic extraction of materials and related properties from scientific articles}

\chapter{Domain adaptation: from superconductors to magnetic materials}


\chapter*{Acknowledgements}
\addcontentsline{toc}{chapter}{Acknowledgements}
\newpage

\addcontentsline{toc}{chapter}{Bibliography}
\bibliographystyle{unsrt}
\bibliography{sample.bib}

\end{document}
