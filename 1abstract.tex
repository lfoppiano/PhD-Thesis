\begin{abstract}

The scientific literature, which contains vast human knowledge, is rapidly expanding, posing challenges in organising and retrieving information. 
Big data methods aid in uncovering patterns and making predictions and have long been applied in chemistry and biology, with other disciplines, such as Materials Science falling behind.
Materials Informatics (MI) is a discipline that leverages computational power to accelerate research in Materials Science by employing techniques like Density Functional Theory (DFT) computations and Machine Learning (ML) for material discovery. 
Despite advances, limited experimental datasets, such as SuperCon, hinder progress. 
SuperCon, a complex database for superconductor materials, faces challenges in consistent data collection. 
Automated processes are needed to extract data from new publications promptly, and manual curation, which is prone to errors, requires careful tools and feature selection.

In this dissertation, we propose an end-to-end pipeline for extracting material information from the scientific literature to improve the efficiency and quality of material databases.
Our work aims to increase automated tasks to reduce the dependency on human intelligence to the minimum. 
The main contributions consist of developing ML-based models for recognising complex material-related expressions from scientific text obtained from PDF documents.  
Material expressions are further processed with a parser that decomposes granular information such as name, formulas, doping, shape, etc. 
The extraction of properties and conditions (e.g., Tc = 3K, 24 GPa, 12 atm) is performed by combining with a general model for extracting measurements of physical quantities, called Grobid-quantities which support the identification and normalisation of units to the SI system. 
We developed SuperMat, a dataset of 164 superconductor articles for evaluation and training. The dataset was constructed and validated by superconductor researchers and provides a structure that contains annotated entities and relations.  
We ran our pipeline on a large set of scientific articles from the Arixv repository collecting a database with over 40000 extracted material-properties records. 
Finally, we proposed a curation interface to validate the extracted data by combining an enhanced PDF viewer and a comprehensive interface. In general, the use of such an interface compared to the traditional manual approach improved precision by 6\% and recall by 47\%. 
The models, dataset, and interface developed in this work will help increase the automation and accuracy of the processing of materials databases such as SuperCon from the scientific literature.
\end{abstract}