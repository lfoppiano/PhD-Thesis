\begin{abstract}

The scientific literature, containing vast human knowledge, is rapidly expanding, posing challenges in organising and retrieving information. 
Big data methods aid in uncovering patterns and making predictions and have been long applied in Chemistry and Biology, with other disciplines, such as Materials Science falling behind.
Materials Informatics (MI) is a discipline that leverages computational power to accelerate research in Materials Science by employing techniques like Density Functional Theory (DFT) computations and Machine Learning (ML) for material discovery. 
Despite advances, limited experimental datasets, like SuperCon, hinder progress. 
SuperCon, a complex database for superconductor materials, faces challenges in consistent data collection. 
Automated processes are needed for timely data extraction from new publications, and manual curation, prone to errors, requires careful tools and feature selection.

In this dissertation, we propose an end-to-end pipeline for extracting material information from scientific literature to improve the efficiency and quality of materials databases.
Our work aims to increase the automated tasks to reduce the dependency on human intelligence to the minimum. 
The main contributions consist of developing ML-based models for recognising complex materials expressions and other information from scientific text that is obtained directly from PDF documents. 
Material expressions are further processed with a parser that decomposes granular information such as name, formulas, doping, shape etc. 
The extraction of properties is performed by combining with a general model for extracting physical quantities and measurements, called Grobid-quantities which support the identification and normalisation of units to the SI system. 
For evaluation and training, we developed SuperMat, a dataset of 142 superconductor articles. The dataset contains both entities and relations between them, and it was validated by domain experts.  
The automatic process collected a database with over 40,000 extracted material-properties records from the related scientific articles from ArXiv. 
Finally, we proposed a curation interface to validate extracted data by combining an enhanced PDF viewer and a simple and sleek interface. Overall, the use of such an interface as compared with the traditional manual approach improved precision by 6\% and recall by 47\%. 

The models, dataset, and interface developed in this work will help increase the automation and accuracy of populating materials databases like SuperCon from the scientific literature.

\end{abstract}